\section*{Appendix A: Abridged Derivation of the EM Algorithm for Gaussian Mixture Models}
\label{app:em_algorithm}

In the Gaussian Mixture Model (GMM) approach, the distribution of samples (vectors) is modelled as:
%
\begin{equation}
	p(\Vec{x} | \Theta) = \sum\nolimits_{m=1}^{M} w_m p(\Vec{x}| \theta_m)
	\label{eqn:mixture_fn}
\end{equation}

\noindent
where $\Vec{x}$ is a $D$-dimensional vector,
$w_m$ is a weight (with constraints $\sum\nolimits_{m=1}^{M} w_m = 1$, $w_m \geq 0$),
and
$p(\Vec{x}| \theta_m)$ is a multivariate Gaussian density function with parameter set $\theta_m = \{ \Vec{\mu}_m, \Mat{\Sigma}_m \}$:
%
\begin{equation}
	p(\Vec{x}| \theta_m) =  {{\mathcal{N}}}( \Vec{x} | \Vec{\mu}_m, \Mat{\Sigma}_m )  = 
		\frac{1}{ (2\pi)^{\frac{D}{2}} | \Mat{\Sigma}_m|^{\frac{1}{2}} }
		\exp \left[ -\frac{1}{2} (\Vec{x}-\Vec{\mu}_m)^T \Mat{\Sigma}_m^{-1} (\Vec{x}-\Vec{\mu}_m) \right]
%
\end{equation}%

\noindent
where $\Vec{\mu}_m$ is the mean vector and $\Mat{\Sigma}_m$ is the covariance matrix.
Thus the complete parameter set for Eqn.~(\ref{eqn:mixture_fn}) is expressed as $\Theta = \{w_m, \theta_m\}_{m=1}^{M}$.
Given a set of training samples, $X=\{\Vec{x}_i\}_{i=1}^{N}$,
we need to find $\Theta$ that suitably models the underlying distribution.
Stated more formally, we need to find $\Theta$ that maximises the following likelihood function:
%
\begin{equation}
	p(X | \Theta) = \prod\nolimits_{i=1}^{N} p(\Vec{x}_i | \Theta)
	\label{eqn:lhood_fn}
\end{equation}

The Expectation-Maximisation (EM) algorithm~\cite{Dempster77, McLachlan-2008, Moon96, Redner84} is an iterative likelihood function optimisation technique,
often used in the pattern recognition and machine learning~\cite{Bishop_2006,Duda01}.
It is a general method for finding the maximum-likelihood estimate of the parameters of an assumed distribution,
when either the training data is incomplete or has missing values, or when the likelihood function can be made analytically tractable
by assuming the existence of (and values for) {\it missing} data.

To apply the EM algorithm to finding $\Theta$, we must first assume that our training data $X$ is incomplete
and assume the existence of missing data $Y = \{y_i\}_{i=1}^{N}$,
where each $y_i$ indicates the mixture component that ``generated'' the corresponding $\Vec{x}_i$.
Thus $y_i \in [1,M] ~ \forall ~ i$ and $y_i = m$ if the $i$-th feature vector ($\Vec{x}_i$) was ``generated'' by the $m$-th component.
If we know the values for $Y$, then Eqn.~(\ref{eqn:lhood_fn}) can be modified to:
%
\begin{equation}
	p(X,Y | \Theta) = \prod\nolimits_{i=1}^{N} w_{y_i} p(\Vec{x}_i| \theta_{y_i})
	\label{eqn:lhood_fn_modified}
\end{equation}

\noindent
As its name suggests, the EM algorithm is comprised of two steps which are iterated: (i)~expectation, followed by (ii)~maximisation.
In the expectation step, the expected value of the complete data log-likelihood, $\log p(X,Y | \Theta)$,
is found with respect to the unknown data $Y = \{y_i\}_{i=1}^{N}$ given training data $X=\{\Vec{x}_i\}_{i=1}^{N}$ and current parameter estimates,
$\Theta^{[k]}$ (where $k$ indicates the iteration number):
%
\begin{equation}
	Q(\Theta, \Theta^{[k]}) = E \left[ \log p(X,Y | \Theta) ~|~ X,\Theta^{[k]} \right]
	\label{eqn:q_fn}
\end{equation}

\noindent
Since $Y$ is a random variable with distribution $p(\mbox{\boldmath $y$}|X,\Theta^{[k]})$, Eqn.~(\ref{eqn:q_fn}) can be written as:
%
\begin{equation}
	Q(\Theta, \Theta^{[k]}) = \int_{\mbox{\boldmath $y$} \in \Upsilon}
								\log p(X,\mbox{\boldmath $y$} | \Theta) ~ p(\mbox{\boldmath $y$}|X,\Theta^{[k]}) ~~ d \mbox{\boldmath $y$}
	\label{eqn:q_fn2}
\end{equation}

\noindent
where \mbox{\boldmath $y$} is an instance of the missing data and $\Upsilon$ is the space of values \mbox{\boldmath $y$} can take on.
The maximisation step then maximises the expectation:
%
\begin{equation}
	\Theta^{[k+1]} = \arg \max_{\Theta} Q(\Theta, \Theta^{[k]})
	\label{eqn:maximize}
\end{equation}

\noindent
The expectation and maximisation steps are iterated until convergence,
or when the increase in likelihood falls below a pre-defined threshold.
As can be seen in Eqn.~(\ref{eqn:q_fn2}), we require $p(\mbox{\boldmath $y$}|X,\Theta^{[k]})$.
We can define it as follows:
%
\begin{equation}
	p(\mbox{\boldmath $y$}|X,\Theta^{[k]}) = \prod\nolimits_{i=1}^{N} p(y_i | \Vec{x}_i, \Theta^{[k]})
\end{equation}

\noindent
Given initial parameters%
%
\footnote{Parameters for $k=0$ can be found via the {\it k}-means algorithm~\cite{Bishop_2006,Duda01,Linde80} (see also Section \ref{sec:param_km}).}
%
~$\Theta^{[k]}$,
we can compute $p(\Vec{x}_i | \theta_m^{[k]})$.
Moreover, we can interpret the mixing weights ($w_m$) as {a-priori} probabilities of each mixture component, ie., $w_m = p(m | \Theta^{[k]})$.
Hence we can apply Bayes' rule~\cite{Duda01} to obtain:
%
\begin{eqnarray}
	p(y_i | \Vec{x}_i, \Theta^{[k]}) & = & \frac{ p(\Vec{x}_i | \theta_{y_i}^{[k]}) p(y_i | \Theta^{[k]}) }{ p(\Vec{x}_i | \Theta^{[k]}) } \\
	~ & = & \frac{ p(\Vec{x}_i | \theta_{y_i}^{[k]}) p(y_i | \Theta^{[k]}) }{ \sum\nolimits_{n=1}^{M} p(\Vec{x}_i | \theta_{n}^{[k]}) p(n | \Theta^{[k]})}
	\label{eqn:p_yi}
\end{eqnarray}

\noindent
Expanding Eqn.~(\ref{eqn:q_fn2}) yields:
%
\begin{eqnarray}
	Q(\Theta, \Theta^{[k]}) & = & \int_{\mbox{\boldmath $y$} \in \Upsilon}
									\log p(X,\mbox{\boldmath $y$} | \Theta) ~ p(\mbox{\boldmath $y$}|X,\Theta^{[k]}) ~~ d \mbox{\boldmath $y$} \\
							~ & = & \sum\nolimits_{\mbox{\boldmath $y$} \in \Upsilon}  \log \prod\nolimits_{i=1}^{N} w_{y_i} p(\Vec{x}_i | \theta_{y_i})  
																		\prod\nolimits_{j=1}^{N} p(y_j | \Vec{x}_j, \Theta^{[k]} ) \\
							~ & = & \sum\nolimits_{y_1=1}^{M} \sum\nolimits_{y_2=1}^{M} \cdots \sum\nolimits_{y_N=1}^{M} 
															\sum\nolimits_{i=1}^{N} \log \left[ w_{y_i} p(\Vec{x}_i | \theta_{y_i}) \right] 
															\prod\nolimits_{j=1}^{N} p(y_j | \Vec{x}_j, \Theta^{[k]} )  \label{eqn:q_fn_expanded}
\end{eqnarray}%

\noindent
It can be shown~\cite{Bilmes98} that Eqn.~(\ref{eqn:q_fn_expanded}) can be simplified to:
%
\begin{eqnarray}
	Q(\Theta, \Theta^{[k]}) & = & \sum\nolimits_{m=1}^{M}  \sum\nolimits_{i=1}^{N} \log[ w_m  p(\Vec{x}_i | \theta_m)] ~ p(m|\Vec{x}_i, \Theta^{[k]}) \\
							~ & = & \sum\nolimits_{m=1}^{M}  \sum\nolimits_{i=1}^{N} \log[ w_m ] ~  p(m|\Vec{x}_i, \Theta^{[k]}) +
									\sum\nolimits_{m=1}^{M}	\sum\nolimits_{i=1}^{N} \log[  p(\Vec{x}_i | \theta_m) ] ~ p(m|\Vec{x}_i, \Theta^{[k]}) ~~~ \\
							~ & = & Q_1 ~~~ + ~~~ Q_2 
\end{eqnarray}%

\noindent
Hence $Q_1$ and $Q_2$ can be maximised separately, to obtain $w_m$ and $\theta_m = \{ \Vec{\mu}_m, \Mat{\Sigma}_m \}$, respectively.
To find the expression which maximises $w_m$, we need to introduce the Lagrange multiplier~\cite{Duda01} $\psi$,
with the constraint $\sum\nolimits_m w_m = 1$, take the derivative of $Q_1$ with respect to $w_m$ and set the result to zero:
%
\begin{eqnarray}
	\frac{\partial Q_1}{\partial w_m} & = & 0 \\
						\therefore ~ 0  & = & \frac{\partial}{\partial w_m}
											\left\{ \sum\nolimits_{m=1}^{M} \sum\nolimits_{i=1}^{N} \log[ w_m ]  ~ p(m|\Vec{x}_i, \Theta^{[k]})
											 		+ \psi \left[ (\sum\nolimits_m w_m) -1 \right] \right\} \\
									~ & = & \sum\nolimits_{i=1}^{N} \frac{1}{w_m} ~ p(m|\Vec{x}_i, \Theta^{[k]}) + \psi \label{eqn:q1_alpha_opt}
\end{eqnarray}

\noindent
Rearranging Eqn.~(\ref{eqn:q1_alpha_opt}) to obtain a value for $\psi$:
%
\begin{eqnarray}
	-\psi w_m & = & \sum\nolimits_{i=1}^{N} p(m|\Vec{x}_i, \Theta^{[k]})
\end{eqnarray}

\noindent
Summing both sides over $m$ yields:
%
\begin{eqnarray}
	-\psi \sum\nolimits_m w_m & = & \sum\nolimits_{i=1}^{N} \sum\nolimits_m p(m|\Vec{x}_i, \Theta^{[k]}) \\
	-\psi 1 & = & \sum\nolimits_{i=1}^{N} 1 \\
	\psi & =  & -N	\label{eqn:lambda_equals_negN}
\end{eqnarray}

\noindent
By substituting Eqn.~(\ref{eqn:lambda_equals_negN}) into Eqn.~(\ref{eqn:q1_alpha_opt}) we obtain:
%
\begin{eqnarray}
			N & = & \sum\nolimits_{i=1}^{N} \frac{1}{w_m} ~ p(m|\Vec{x}_i, \Theta^{[k]}) \\
	\therefore ~ w_m & = & \frac{1}{N} \sum\nolimits_{i=1}^{N} p(m|\Vec{x}_i, \Theta^{[k]}) 
	\label{eqn:alpha_m_solved}
\end{eqnarray}

\noindent
To find expressions which maximise $\Vec{\mu}_m$ and $\Mat{\Sigma}_m$, let us now expand $Q_2$:
%
\begin{eqnarray}
	Q_2 & = &   \sum\nolimits_{m=1}^{M}	\sum\nolimits_{i=1}^{N} \log[  p(\Vec{x}_i | \theta_m) ] ~ p(m|\Vec{x}_i, \Theta^{[k]}) 	\\
	 ~  & = &   \sum\nolimits_{m=1}^{M} \sum\nolimits_{i=1}^{N} \left[
					- \frac{1}{2} \log(|\Mat{\Sigma}_m|) - \frac{1}{2} (\Vec{x}_i - \Vec{\mu}_m)^T \Mat{\Sigma}_m^{-1} (\Vec{x}_i - \Vec{\mu}_m)
					\right] p(m|\Vec{x}_i, \Theta^{[k]})  \label{eqn:q2_expanded}
\end{eqnarray}

\noindent
where $-\frac{D}{2} \log(2\pi)$ was omitted since it vanishes when taking a derivative with respect to $\Vec{\mu}_m$ or $\Mat{\Sigma}^{-1}_m$.
To find the expression which maximises $\Vec{\mu}_m$, we need to take the derivative of $Q_2$ with respect to $\Vec{\mu}_m$, and set the result to zero:
%
\begin{eqnarray}
	\frac{\partial Q_2}{\partial \Vec{\mu}_m} & = & 0   \\
 0 & = & \frac{\partial}{\partial \Vec{\mu}_m} \left\{ \sum\nolimits_{m=1}^{M} \sum\nolimits_{i=1}^{N} \left[
					- \frac{1}{2} \log(|\Sigma_m|) - \frac{1}{2} (\Vec{x}_i - \Vec{\mu}_m)^T \Mat{\Sigma}_m^{-1} (\Vec{x}_i - \Vec{\mu}_m)
					\right] p(m|\Vec{x}_i, \Theta^{[k]}) \right\} ~~~ ~~~ ~~~  \label{eqn:q2_mu_partone}
\end{eqnarray}%

\noindent
L\"{u}tkepohl~\cite{Lutkepohl96} states that
\mbox{$\frac{ \partial \Vec{z}^T \Mat{A} \Vec{z} }{ \partial \Vec{z} } =  (\Mat{A} + \Mat{A}^T)\Vec{z}$},
~ $(\Mat{A}^{-1})^T = (\Mat{A}^T)^{-1}$ and if $\Mat{A}$ is symmetric, then $\Mat{A} = \Mat{A}^T$.
Since $\Mat{\Sigma}_m$ is symmetric, Eqn.~(\ref{eqn:q2_mu_partone}) reduces to:
%
\begin{eqnarray}
	0 & = & \sum\nolimits_{i=1}^{N} - \frac{1}{2} 2 \Mat{\Sigma}_m^{-1} (\Vec{x}_i - \Vec{\mu}_m) p(m|\Vec{x}_i, \Theta^{[k]}) \\
	~ & = & \sum\nolimits_{i=1}^{N} \left[ - \Mat{\Sigma}_m^{-1} \Vec{x}_i p(m|\Vec{x}_i, \Theta^{[k]}) + \Mat{\Sigma}_m^{-1} \Vec{\mu}_m p(m|\Vec{x}_i, \Theta^{[k]})  \right] ~~~ \\
	\therefore ~ \sum\nolimits_{i=1}^{N} \Mat{\Sigma}_m^{-1} \Vec{\mu}_m p(m|\Vec{x}_i, \Theta^{[k]}) & = & \sum\nolimits_{i=1}^{N} \Mat{\Sigma}_m^{-1} \Vec{x}_i p(m|\Vec{x}_i, \Theta^{[k]}) 
\end{eqnarray}%

\noindent
Multiplying both sides by $\Mat{\Sigma}_m$ yields:
%
\begin{eqnarray}
	\sum\nolimits_{i=1}^{N} \Vec{\mu}_m p(m|\Vec{x}_i, \Theta^{[k]}) & = & \sum\nolimits_{i=1}^{N} \Vec{x}_i p(m|\Vec{x}_i, \Theta^{[k]})  \\
\therefore ~ \Vec{\mu}_m & = & \frac{ \sum\nolimits_{i=1}^{N} \Vec{x}_i p(m|\Vec{x}_i, \Theta^{[k]}) }
								   { \sum\nolimits_{i=1}^{N} p(m|\Vec{x}_i, \Theta^{[k]}) }
\end{eqnarray}

\noindent
L\"{u}tkepohl~\cite{Lutkepohl96} states that:  
$|\Mat{A}^{-1}| = |\Mat{A}|^{-1}$ 
and 
$\mbox{tr}(\Mat{A}\Mat{B}) = \mbox{tr}(\Mat{B}\Mat{A})$.
Since $\mbox{tr}{(\Vec{z}A\Vec{z}^T)} = \mbox{tr}(\mbox{scalar})$,
we can rewrite Eqn.~(\ref{eqn:q2_expanded}) as:
%
\begin{eqnarray}
	Q_2 & = &   \sum\nolimits_{m=1}^{M} \sum\nolimits_{i=1}^{N} \left[
					\frac{1}{2} \log(|\Mat{\Sigma}_m^{-1}|) - \frac{1}{2} \mbox{tr}(\Mat{\Sigma}_m^{-1} (\Vec{x}_i - \Vec{\mu}_m) (\Vec{x}_i - \Vec{\mu}_m)^T)
					\right] p(m|\Vec{x}_i, \Theta^{[k]})  \label{eqn:q2_rewritten}
\end{eqnarray}

\noindent
According to L\"{u}tkepohl~\cite{Lutkepohl96},
$\frac{\partial \log(|\Mat{A}|)}{\partial \Mat{A}} = (\Mat{A}^{T})^{-1}$
and $\frac{\partial \mbox{tr}(\Mat{B}\Mat{A})}{\partial \Mat{B}} = \Mat{A}^T$.
Moreover, we note that $\Vec{z}\Vec{z}^T$ is a symmetric matrix.
To find an expression which maximises~$\Mat{\Sigma}_m$,
we can take the derivative of Eqn.~(\ref{eqn:q2_rewritten}) with respect to $\Mat{\Sigma}_m^{-1}$
and set the result to zero:
%
\begin{eqnarray}
	0 & = & \frac{\partial Q_2}{\partial \Mat{\Sigma}_m^{-1}}  \\
	~ & = & \frac{\partial}{\partial  \Mat{\Sigma}_m^{-1}} \left\{
		\sum\nolimits_{m=1}^{M} \sum\nolimits_{i=1}^{N} \left[
					\frac{1}{2} \log(|\Mat{\Sigma}_m^{-1}|) - \frac{1}{2} \mbox{tr}\left(\Mat{\Sigma}_m^{-1} (\Vec{x}_i - \Vec{\mu}_m) (\Vec{x}_i - \Vec{\mu}_m)^T \right)
					\right] p(m|\Vec{x}_i, \Theta^{[k]})
	\right\} \\
	~ & = & \sum\nolimits_{i=1}^{N} \left[ \frac{1}{2} \Mat{\Sigma}_m - \frac{1}{2} (\Vec{x}_i - \Vec{\mu}_m) (\Vec{x}_i - \Vec{\mu}_m)^T \right] p(m|\Vec{x}_i, \Theta^{[k]}) \\
\end{eqnarray}%

\noindent
thus
%
\begin{eqnarray}
	\frac{1}{2} \Mat{\Sigma}_m \sum\nolimits_{i=1}^{N} p(m|\Vec{x}_i, \Theta^{[k]}) & = & \frac{1}{2} \sum\nolimits_{i=1}^{N} 
																		(\Vec{x}_i - \Vec{\mu}_m) (\Vec{x}_i - \Vec{\mu}_m)^T p(m|\Vec{x}_i, \Theta^{[k]}) \\
	\therefore ~ \Mat{\Sigma}_m & = & \frac{\sum\nolimits_{i=1}^{N} (\Vec{x}_i - \Vec{\mu}_m) (\Vec{x}_i - \Vec{\mu}_m)^T p(m|\Vec{x}_i, \Theta^{[k]})}
						{\sum\nolimits_{i=1}^{N} p(m|\Vec{x}_i, \Theta^{[k]})}
\end{eqnarray}
%\end{small}

\noindent
In summary,
%
\begin{eqnarray}
	w_m^{[k+1]}	& = & \frac{1}{N} \sum\nolimits_{i=1}^{N} p(m|\Vec{x}_i, \Theta^{[k]})  \label{eqn:copy_of_solved_alpha}  \\  
	\Vec{\mu}_m^{[k+1]} & = & \frac{ \sum\nolimits_{i=1}^{N} \Vec{x}_i ~ p(m|\Vec{x}_i, \Theta^{[k]}) }
								   { \sum\nolimits_{i=1}^{N} p(m|\Vec{x}_i, \Theta^{[k]}) }  \label{eqn:copy_of_solved_mu} \\
	\Mat{\Sigma}_m^{[k+1]}	& = & \frac{\sum\nolimits_{i=1}^{N} (\Vec{x}_i - \Vec{\mu}_m^{[k+1]}) (\Vec{x}_i - \Vec{\mu}_m^{[k+1]})^T p(m|\Vec{x}_i, \Theta^{[k]})}
						{\sum\nolimits_{i=1}^{N} p(m|\Vec{x}_i, \Theta^{[k]})}  \label{eqn:copy_of_solved_sigma}
\end{eqnarray}

\noindent
where
%
\begin{equation}
	p(m | \Vec{x}_i, \Theta^{[k]}) = \frac{ p(\Vec{x}_i | \theta_{m}^{[k]}) p(m | \Theta^{[k]}) }{ \sum\nolimits_{n=1}^{M} p(\Vec{x}_i | \theta_{n}^{[k]}) p(n | \Theta^{[k]})}
\end{equation}
%
which can be explicitly stated as:
%
\begin{equation}
	p(m | \Vec{x}_i, \Theta^{[k]}) = \frac{ {\mathcal{N}}( \Vec{x}_i | \Vec{\mu}_m^{[k]},  \Mat{\Sigma}_m^{[k]} )  w_m^{[k]} }
										{ \sum\nolimits_{n=1}^{M} {\mathcal{N}}( \Vec{x}_i | \Vec{\mu}_n^{[k]}, \Mat{\Sigma}_n^{[k]} ) w_n^{[k]} }
\end{equation}

\noindent
If we let $l_{m,i} = p(m | \Vec{x}_i, \Theta^{[k]})$ and $L_m = \sum\nolimits_{i=1}^{N} l_{m,i} $, we can restate Eqns.~(\ref{eqn:copy_of_solved_alpha})
to~(\ref{eqn:copy_of_solved_sigma}) as:
%
\begin{eqnarray}
	w_m^{[k+1]}	& = & \frac{L_m}{N} \\  
	\Vec{\mu}_m^{[k+1]} & = & \frac{1}{L_m} \sum\nolimits_{i=1}^{N} \Vec{x}_i ~ l_{m,i} \\
	\Mat{\Sigma}_m^{[k+1]}	& = & \frac{1}{L_m} \sum\nolimits_{i=1}^{N} (\Vec{x}_i - \Vec{\mu}_m^{[k+1]}) (\Vec{x}_i - \Vec{\mu}_m^{[k+1]})^T  l_{m,i}
\end{eqnarray}
