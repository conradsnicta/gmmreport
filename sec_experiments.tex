\section{Evaluation}
\label{sec:eval}

\subsection{Speedup from Multi-Threading}

\begin{table*}
\centering
\small
\begin{tabular}{|l|c|c|c|c|c|c|c|}
\hline
{\bf dataset} & {\bf $N_v$} & {\bf dims} & {\bf $N_g$} & {\bf MLPACK fit time} & {\bfseries {\it\bfseries gmm\_diag} fit time} & {\bf MLPACK $\log(p(X|\lambda))$} & {\bf {\it\bfseries gmm\_diag} $\log(p(X|\lambda))$} \\
\hline
cloud & 2048 & 10 & 5 & ~~~~1.375s & ~~{\bf 0.164s} & $-59.9 \times 10^{3}$ & $-63.0 \times 10^{3}$ \\
ozone & 2534 & 72 & 6 & ~~~~1.982s & ~~{\bf 0.202s} & $-230 \times 10^{3}$ & $-399 \times 10^{3}$ \\
winequality & 6497 & 11 & 30 & ~~~18.201s & ~~{\bf 1.257s} & $-47.5 \times 10^{3}$ & $-15.6 \times 10^{3}$ \\
corel & 37749 & 32 & 50 & ~~501.601s & {\bf 22.016s} & $~~2.99 \times 10^{6}$ & $~~2.89 \times 10^{6}$ \\
cup98b & 95413 & 56 & 30 & 1207.86s & {\bf 48.425s} & $-11.9 \times 10^{6}$ & $-6.62 \times 10^{6}$ \\
birch3 & 100000 & 2 & 6 & \\
phy & 150000 & 78 & 30 & & {\bf 78.44s} & & -7.94M \\
covertype & 581012 & 55 & 21 & \\
pokerhand & 1000000 & 10 & 25 & 30 & \\
\hline
\end{tabular}
\caption{Datasets used for comparisons with full-covariance GMM estimation.}
\label{tab:results}
\end{table*}

% \begin{table*}
% \begin{center}
% \begin{tabular}{|l|c|c|c|c|c|c|c|}
% \hline
% {\bf dataset} & {\bf $N_v$} & {\bf dims} & {\bf $N_g$} & {\bf full-covariance fit time} & {\bf Armadillo fit time} & {\bf full-diagonal $\log(p(X|\lambda))$} & {\bf Armadillo $\log(p(X|\lambda))$} \\
% \hline
% cloud & 2048 & 10 & 5 & 1.375s & {\bf 0.164s} & -59.9k & -63.0k \\
% ozone & 2534 & 72 & 6 & 1.982s & {\bf 0.202s} & -230k & -399k \\
% winequality & 6497 & 11 & 30 & 18.201s & {\bf 1.257s} & -47.5k & -15.6k
% \\
% corel & 37749 & 32 & 50 & 501.601s & {\bf 22.016s} & 2.99M & 2.89M \\
% cup98b & 95413 & 56 & 30 & 1207.86s & {\bf 48.425s} & -11.9M & -6.62M \\
% birch3 & 100000 & 2 & 6 & \\
% phy & 150000 & 78 & 30 & & {\bf 78.44s} & & -7.94M \\
% covertype & 581012 & 55 & 21 & \\
% pokerhand & 1000000 & 10 & 25 & 30 & \\
% \hline
% \end{tabular}
% \end{center}
% \caption{Datasets used for comparisons with full-covariance GMM estimation.}
% \label{tab:results}
% \end{table*}

% \begin{table*}
% \begin{center}
% \begin{tabular}{|l|c|c|c|c|c|c|c|}
% \hline
% {\bf dataset} & {\bf $N_v$} & {\bf dims} & {\bf $N_g$} & {\bf
% full-covariance fit time} & {\bf Armadillo fit time} & {\bf full-diagonal
% $\log(p(X|\lambda))$} & {\bf Armadillo $\log(p(X|\lambda))$} \\
% \hline
% cloud & 2048 & 10 & 5 & 1.375s & {\bf 0.164s} & -59.9k & -63.0k \\
% ozone & 2534 & 72 & 6 & 1.982s & {\bf 0.202s} & -230k & -399k \\
% winequality & 6497 & 11 & 30 & 18.201s & {\bf 1.257s} & -47.5k & -15.6k \\
% corel & 37749 & 32 & 50 & 501.601s & {\bf 22.016s} & 2.99M & 2.89M \\
% cup98b & 95413 & 56 & 30 & & {\bf 48.425s} &  & -6.62M \\
% birch3 & 100000 & 2 & 6 & \\
% phy & 150000 & 78 & 30 & \\
% covertype & 581012 & 55 & 21 & \\
% pokerhand & 1000000 & 10 & 25 & 30 & \\
% \hline
% \end{tabular}
% \end{center}
% \caption{Datasets used for comparisons with full-covariance GMM estimation.}
% \label{tab:results}
% \end{table*}

To demonstrate the achievable speedup with the multi-threaded versions of the EM and {\it k}-means algorithms,
we trained a GMM with 100 Gaussians on a recent 16 core machine using a synthetic dataset comprising 1,000,000 samples with 100 dimensions.
10 iterations of the {\it k}-means algorithm and 10 iterations of the EM algorithm were used.
The samples were stored in double precision floating point format, resulting in a total data size of approximately 762~Mb.
% The corresponding source code is shown in Figure~\ref{fig:timing_prog}.

Figure~\ref{fig:speedup} shows that a speedup of an order of magnitude is achieved when all 16 cores are used.
Specifically, for the synthetic dataset used in this demonstration,
the training time was reduced from approximately 340 seconds to about 33 seconds.
In each case, the {\it k}-means algorithm took approximately 30\% of the total training time.
%which, in other words, indicates that {\it k}-means executed about 2 times faster than the EM algorithm.

We note that the overall speedup is below the idealised linear speedup.
This is likely due to overheads related to OpenMP and reduction operations described in Section~\ref{sec:param_em_parallel},
as well as memory access contention (stemming from concurrent access to memory by multiple cores)~\cite{McCool_2012}.

\subsection{Comparison with Full-Covariance GMMs}

% I guess we should move the parallelism experiment here, if you go with this
% approach.

% This paragraph could be reworked---it doesn't really fit with the "flow" of
% the paper.  But I think it would be nice to show that using a diagonal GMM
% doesn't cost us too much---because that would be my primary concern as a
% reviewer (i.e. "this contribution doesn't matter because nobody uses diagonal
% GMMs").
In order to validate our intuition that a diagonal GMM is a good choice instead
of the significantly more complex problem of estimating GMMs with full
covariance matrices, we compare the {\it gmm\_diag} class (described in Section~\ref{sec:implementation})
against the full-covariance GMM implementation in the MLPACK C++ machine learning library~\cite{Curtin_2013}.

We select several common datasets from the UCI machine learning dataset
repository~\cite{Lichman_2013}, and train both diagonal and full-covariance
GMMs on these datasets.  The number of Gaussians in the model was chosen
according to the original source of the dataset; where possible, 3 times the
number of classes in the dataset was used.  In some cases, small amounts of
Gaussian noise needed to be added to the dataset for stability.
% It takes too long to do BIC on this, we don't have time.  We could handle it
% after reviews possibly?  Really the number of Gaussians does not matter
% anyway, we only care here about how long it takes.
These simulations were run on a standard 8-core desktop; both implementations
used all 8 cores during training, with a maximum of 250 iterations of the EM
algorithm before the fitting procedure was terminated.  The fitting procedure
was run 10 times, and the best log-likelihood of those 10 runs are given, as
well as the average runtime for the fitting.

The results are given in Table \ref{tab:results}, along with the dataset
information (number of points, number of dimensions, and number of Gaussians
used for modeling).  We can see that the diagonal GMM implementation in 
the {\it gmm\_diag} class provides order-of-magnitude or more speedup over the full covariance
implementation in MLPACK, without significant loss in goodness-of-fit (as measured by log-likelihood).
In two cases ({\it winequality} and {\it cup98b}) the goodness-of-fit is notably higher for the {\it gmm\_diag} class.

Additionally, we perform a more detailed investigation of the behavior of each
implementation on the {\tt corel} dataset.  In order to verify the results of
Reynolds~\cite{Reynolds_2000}, we train diagonal and full-covariance GMMs on a
variety of choices of $N_g$, collecting the log-likelihood of the trained model
at each EM iteration.  This shows us how quickly each model is able to fit to
the data.  Figure~\ref{fig:lc} shows the resulting learning curve, demonstrating
that the diagonal GMM implementation is able to converge to a better
log-likelihood in less time than the full-covariance implementation.

\begin{figure*}
\begin{center}
% This figure is horrible.  I have not thought of a better way to plot this
% here.  But maybe it is a start.  Like for real this is the ugliest graph I
% have ever produced in my life.
% \includegraphics[width=\textwidth]{data/garbage_fire.png}
\end{center}
\caption{GMM training time vs. log-likelihood of model for diagonal GMMs (blue)
and non-diagonal GMMs (red).  The simulation is performed for many values of
$N_g$, showing that for the {\tt corel} dataset, we can get a faster and better
fit with our implementation of diagonal GMMs than with a full-covariance GMM,
even when the diagonal GMM must have a greater $N_g$ to achieve a better fit.}
\label{fig:lc}
\end{figure*}
