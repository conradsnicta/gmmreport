\section{Introduction}

Modelling multivariate data through a convex mixture of Gaussians, also known as a Gaussian mixture model (GMM),
has many uses in fields such as signal processing, econometrics, pattern recognition, machine learning and computer vision.
Examples of applications include
multi-stage feature extraction for action recognition~\cite{Carvajal_2016a},
modelling of intermediate features~\cite{Ge_ICIP_2015} derived from deep convolutional neural networks~\cite{Ge_2016,LeCun_Nature_2015},
classification of human epithelial cell images~\cite{Wiliem_PR_2014},
implicit sparse coding for face recognition~\cite{Wong_2014},
and probabilistic foreground estimation for surveillance systems~\cite{Reddy_2013}.
GMMs have also been successfully used for speaker
verification~\cite{Reynolds_2000} and GMMs are commonly used as the emission
distribution for Hidden Markov models~\cite{Bilmes98}.

In the GMM approach, a distribution of samples (vectors) is modelled as:
%
\begin{equation}
  p(\Vec{x} | \lambda) = \sum\nolimits_{g=1}^{N_G} w_g ~ {{\mathcal{N}}}( \Vec{x} ~|~ \Vec{\mu}_g, \Mat{\Sigma}_g )
  \label{eqn:gmm_prob}
\end{equation}%
%
where $\Vec{x}$ is a $D$-dimensional vector,
$w_g$ is the weight for component $g$ (with constraints $\sum\nolimits_{g=1}^{N_G} w_g = 1$, $w_g \geq 0$),
and
${{\mathcal{N}}}( \Vec{x} | \Vec{\mu}, \Mat{\Sigma})$ is a $D$-dimensional Gaussian density function with mean $\Vec{\mu}$ and covariance matrix $\Mat{\Sigma}$:
%
\begin{equation}
  {{\mathcal{N}}}( \Vec{x} ~|~ \Vec{\mu}, \Mat{\Sigma} )  = 
  \frac{1}{ (2\pi)^{\frac{D}{2}} | \Mat{\Sigma}|^{\frac{1}{2}} }
  \exp \left[ -\frac{1}{2} (\Vec{x}-\Vec{\mu})^\top \Mat{\Sigma}^{-1} (\Vec{x}-\Vec{\mu}) \right]
  \label{eqn:gaussian}
\end{equation}%
%
where $|\Mat{\Sigma}|$ and $\Mat{\Sigma}^{-1}$ denote the determinant and inverse of $\Mat{\Sigma}$, respectively,
while $\Vec{x}^\top$ denotes the transpose of $\Vec{x}$.
The full parameter set can be compactly stated as $\lambda = \{ w_g, \Vec{\mu}_g, \Mat{\Sigma}_g \}_{g=1}^{N_G}$,
where $N_G$ is the number of Gaussians.

Given a training dataset and a value for $N_G$,
the estimation of $\lambda$ is typically done through a
tailored instance of Expectation Maximisation (EM) algorithm~\cite{Dempster77, McLachlan-2008, Moon96, Redner84}.
The {\it k}-means algorithm~\cite{Bishop_2006,Duda01,Linde80} is also typically used for providing the initial estimate of $\lambda$ for the EM algorithm.
Choosing the optimal $N_G$ is data dependent and beyond the scope of this work; see~\cite{Hamerly_2003,Pelleg_2000} for example methods.

%The {\it k}-means and EM algorithms are computationally intensive.
Unfortunately, GMM parameter estimation via the EM algorithm is computationally
intensive, and % <- talking up the difficulty of the problem a little bit
given the ever growing sizes of datasets and the need for fast \& accurate modelling of such datasets,
we have provided a readily accessible and open source implementation of multi-threaded (parallelised) versions 
of the {\it k}-means and EM algorithms.
The implementation is provided as a user-friendly class as part of the Armadillo C++ linear algebra library, which is
licensed under the permissive Apache~2.0 license~\cite{Laurent_2008},
thereby allowing unencumbered use in commercial products.

We continue as follows.
In Section~\ref{sec:param_em} we provide an overview of parameter estimation via the EM algorithm,
its reformulation for multi-threaded execution,
and approaches for improving numerical stability.
In Section~\ref{sec:param_km} we provide a summary of the {\it k}-means algorithm
along with approaches for improving its convergence and modelling accuracy.
The numerical implementation in C++ is overviewed in~Section~\ref{sec:implementation},
where we list the user accessible functions
and provide a demonstration of the achievable training speedup due to multi-threading.

% Parallelisation is achieved through refactoring the original EM and {\it k}-means algorithms
% into a MapReduce-like framework~\cite{MapReduce_2004} and employing OpenMP compiler directives~\cite{OpenMP_2007}.



