\section{Introduction}

Modelling multivariate data through mixtures of Gaussians, also known as Gaussian Mixture Models (GMMs),
has many uses in fields such as statistics, econometrics, pattern recognition, machine learning and computer vision.
Examples of applications include
multi-stage feature extraction for action recognition~\cite{Carvajal_2016a},
modelling of intermediate features~\cite{Ge_ICIP_2015} derived from deep convolutional neural networks~\cite{Ge_2016,LeCun_Nature_2015},
classification of human epithelial cell images~\cite{Wiliem_PR_2014},
implicit sparse coding for face recognition~\cite{Wong_2014},
and probabilistic foreground estimation for surveillance systems~\cite{Reddy_2013}.

In the GMM approach, a distribution of samples (vectors) is modelled as:
%
\begin{equation}
  p(\Vec{x} | \lambda) = \sum\nolimits_{g=1}^{N_G} w_g ~ {{\mathcal{N}}}( \Vec{x} ~|~ \Vec{\mu}_g, \Mat{\Sigma}_g )
  \label{eqn:gmm_prob}
\end{equation}%
%
where $\Vec{x}$ is a $D$-dimensional vector,
$w_g$ is the weight for component $g$ (with constraints $\sum\nolimits_{g=1}^{N_G} w_g = 1$, $w_g \geq 0$),
and
${{\mathcal{N}}}( \Vec{x} | \Vec{\mu}, \Mat{\Sigma})$ is a $D$-dimensional Gaussian density function with mean $\Vec{\mu}$ and covariance matrix $\Mat{\Sigma}$:
%
\begin{equation}
  {{\mathcal{N}}}( \Vec{x} | \Vec{\mu}, \Mat{\Sigma} )  = 
  \frac{1}{ (2\pi)^{\frac{D}{2}} | \Mat{\Sigma}|^{\frac{1}{2}} }
  \exp \left[ -\frac{1}{2} (\Vec{x}-\Vec{\mu})^\top \Mat{\Sigma}^{-1} (\Vec{x}-\Vec{\mu}) \right]
  \label{eqn:gaussian}
\end{equation}%
%
The full parameter set can be compactly stated as $\lambda = \{ w_g, \Vec{\mu}_g, \Mat{\Sigma}_g \}_{g=1}^{N_G}$,
where $N_G$ is the number of Gaussians.



TODO: choice for $\Mat{\Sigma}$: full or diagonal.
effects: fewer free parameters; much simpler matrix inverse; simpler computation of the determinant.
can still model correlations in data through the use of several gaussians -- ref to old Reynolds paper?

TODO: mention methods for selecting  $N_G$ ?  

Choosing the optimal $N_G$ is typically data dependent and beyond the scope of this work; see~\cite{Hamerly_2003,Pelleg_2000} for example methods.

TODO: overview of all the sections
