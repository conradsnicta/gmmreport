\section{Introduction}

Modelling multivariate data through a convex mixture of Gaussians, also known as a Gaussian Mixture Model (GMM),
has many uses in fields such as statistics, econometrics, pattern recognition, machine learning and computer vision.
Examples of applications include
multi-stage feature extraction for action recognition~\cite{Carvajal_2016a},
modelling of intermediate features~\cite{Ge_ICIP_2015} derived from deep convolutional neural networks~\cite{Ge_2016,LeCun_Nature_2015},
classification of human epithelial cell images~\cite{Wiliem_PR_2014},
implicit sparse coding for face recognition~\cite{Wong_2014},
and probabilistic foreground estimation for surveillance systems~\cite{Reddy_2013}.

In the GMM approach, a distribution of samples (vectors) is modelled as:
%
\begin{equation}
  p(\Vec{x} | \lambda) = \sum\nolimits_{g=1}^{N_G} w_g ~ {{\mathcal{N}}}( \Vec{x} ~|~ \Vec{\mu}_g, \Mat{\Sigma}_g )
  \label{eqn:gmm_prob}
\end{equation}%
%
where $\Vec{x}$ is a $D$-dimensional vector,
$w_g$ is the weight for component $g$ (with constraints $\sum\nolimits_{g=1}^{N_G} w_g = 1$, $w_g \geq 0$),
and
${{\mathcal{N}}}( \Vec{x} | \Vec{\mu}, \Mat{\Sigma})$ is a $D$-dimensional Gaussian density function with mean $\Vec{\mu}$ and covariance matrix $\Mat{\Sigma}$:
%
\begin{equation}
  {{\mathcal{N}}}( \Vec{x} | \Vec{\mu}, \Mat{\Sigma} )  = 
  \frac{1}{ (2\pi)^{\frac{D}{2}} | \Mat{\Sigma}|^{\frac{1}{2}} }
  \exp \left[ -\frac{1}{2} (\Vec{x}-\Vec{\mu})^\top \Mat{\Sigma}^{-1} (\Vec{x}-\Vec{\mu}) \right]
  \label{eqn:gaussian}
\end{equation}%
%
where $|\Mat{\Sigma}|$ and $\Mat{\Sigma}^{-1}$ denote the determinant and inverse of $\Mat{\Sigma}$, respectively,
while $\Vec{x}^\top$ denotes the transpose of $\Vec{x}$.
The full parameter set can be compactly stated as $\lambda = \{ w_g, \Vec{\mu}_g, \Mat{\Sigma}_g \}_{g=1}^{N_G}$,
where $N_G$ is the number of Gaussians.
Choosing the optimal $N_G$ is typically data dependent and beyond the scope of this work; see~\cite{Hamerly_2003,Pelleg_2000} for example methods.

% TODO: choice for $\Mat{\Sigma}$: full or diagonal.
% effects: fewer free parameters; much simpler matrix inverse; simpler computation of the determinant.
% can still model correlations in data through the use of several gaussians -- ref to old Reynolds paper?

There are two main choices for the type of covariance matrix $\Mat{\Sigma}$: full or diagonal.
% The values on the main diagonal (diagonal elements) are variances for each dimension,
% while the off diagonal elements are the covariances across dimensions.
While full covariance matrices have more capacity for modelling data,
diagonal covariance matrices provide several practical advantages:
{\bf (i)}~the computationally expensive (and potentially unstable) matrix inverse operation is reduced to simply to taking the reciprocals of the diagonal elements,
{\bf (ii)}~the determinant operation is considerably simplified to taking the product of the diagonal elements,
{\bf (iii)}~diagonal covariance matrices contain fewer parameters that need to be estimated, and hence require fewer training samples~\cite{Duda01}.
Furthermore, diagonal covariance GMMs with $N_G > 1$ can model distributions of samples with correlated elements,
which in turn suggests that full covariance GMMs can be approximated using a diagonal covariance GMM with a larger number of Gaussians~\cite{Reynolds_2000}.
% It has also been empirically observed that diagonal covariance GMMs outperform full covariance GMMs~\cite{Reynolds95, Reynolds95b, Reynolds00}.


Parallelisation is achieved through refactoring the original EM and {\it k}-means algorithms
into a MapReduce-like framework~\cite{MapReduce_2004} and employing OpenMP compiler directives~\cite{OpenMP_2007}.
% which can be easily implemented with the aid of OpenMP compiler directives~\cite{OpenMP_2007}.

TODO: overview of all the sections

